
\documentclass[11pt]{article}
\usepackage{geometry}
\geometry{letterpaper}
\usepackage{amsmath} % for mathematical features
\usepackage{enumitem} % for customizing lists
\title{Introductory Macroeconomics for Engineers}
\author{Martin A. Valdez}
\date{IE 1}

\begin{document}
\maketitle
\subsection*{Exercice 14}
Travaillons sur le modèle de consommation-épargne à deux périodes que nous avons discuté en classe.
Dans ce problème, nous avons un agent représentatif qui vit pendant deux périodes.
Chaque période, l'agent reçoit une dotation de \(Y_1\) et \(Y_2\), respectivement.
L'agent peut consommer \(C_1\) et \(C_2\) à chaque période, respectivement.
L'agent a la possibilité d'épargner pendant la première période.
Ces économies sont investies, ce qui génère un rendement de \(r\) pendant la deuxième période.
L'agent est impatient, donc il actualise la consommation future avec un facteur d'actualisation \(\beta \in (0,1)\).

Le problème de l'agent est de maximiser la fonction d'utilité :
\begin{align*}
    \max_{ C_1, C_2} U = u(C_1) + \beta u(C_2)
\end{align*}
Sous la contrainte budgétaire de chaque période :
\begin{align*}
    & C_1 + S = Y_1
    \\
    & C_2 = Y_2 + (1 + r)S.
\end{align*}

À la fin de la deuxième période, après avoir consommé \(C_2\), l'agent meurt.

\begin{enumerate}
    \item Combinez les deux contraintes budgétaires pour former une seule contrainte budgétaire \textit{intertemporelle}.
    \item Remplacez la contrainte budgétaire intertemporelle dans la fonction d'utilité pour simplifier le problème.
    \item Calculez les conditions de premier ordre pour le problème. Ceci est l'\textbf{équation d'Euler}.
    \item Supposons que la fonction d'utilité soit donnée par \(u(C) = \ln(C)\). Remplacez cette fonction d'utilité dans l'équation d'Euler.
    \item Utilisez cette équation pour résoudre le niveau optimal de consommation présente, \(C_1\).
    \item Que se passe-t-il au niveau optimal de la consommation présente si le taux d'intérêt augmente ? Pourquoi ?
    \item Que se passe-t-il au niveau optimal de la consommation présente si l'agent devient plus impatient 
    (c'est-à-dire, \(\beta\) diminue) ? Et s'il devient plus patient ? Pourquoi ?
    \item Supposons que la dotation de l'agent pendant la première période augmente. Que se passe-t-il au niveau optimal de la consommation présente ? Pourquoi ?
    Que se passe-t-il au niveau optimal de la consommation future ? Pourquoi ? Que se passe-t-il si la dotation pendant la deuxième période augmente ?
    \item Résolvez le niveau optimal d'épargne, \(S\). Que se passe-t-il à l'épargne si le taux d'intérêt augmente ? Pourquoi ?
\end{enumerate}

Faisons maintenant le lien avec le modèle de Solow. Supposons que nous sommes à l'état stationnaire du modèle de Solow, de sorte que
\(c_1 = c_2 = c^*\), et que l'économie est à l'état stationnaire, de sorte que \(y_1 = y_2 = y^*\).

\begin{enumerate}[resume]
    \item Montrez que l'équation d'Euler peut être écrite comme :
    \begin{align*}
        \frac{1}{c^*} =
        \beta(1 + r) \frac{1}{c^*} \implies 1 = \beta(1 + r).
    \end{align*} 
\end{enumerate}

Nous appelons \(\beta\) un \textit{paramètre de préférence}, qui décrit une propriété \textit{intrinsèque} des préférences de l'agent.
En revanche, \(r\) peut être considéré comme une \textit{quantité d'équilibre}, qui est déterminée par le marché.
Si l'équation précédente n'est pas satisfaite, l'agent ajustera sa consommation pour la satisfaire en épargnant ou en désépargnant.


\end{document}