\documentclass{beamer}
% Load csquotes for the \enquote command
\usepackage{csquotes}

% Load biblatex for bibliography management
\usepackage[backend=biber, style=authoryear]{biblatex}
\addbibresource{bibliography.bib} % Specify your BibTeX bibliography file

\mode<presentation> {
\usetheme{CambridgeUS}
\usecolortheme{seahorse}
}

\title{Macroéconomie 1}
\author{Mart\'in Valdez}
\date{IE1}

\begin{document}

\begin{frame}
\titlepage
\end{frame}

\begin{frame}
\frametitle{Overview}
\tableofcontents
\end{frame}

% Slide Template
\section{Introduction}
\begin{frame}
\frametitle{Introduction}
\begin{itemize}
    \item Course Overview
    \item Objectives
    \item Grading
\end{itemize}
\end{frame}

\section{Economic Concepts}
\begin{frame}
\frametitle{Micro vs. Macro}
\begin{itemize}
    \item \textbf{Microéconomie}
    L'étude des agents économiques individuels tels que les ménages et les entreprises,
    comment ils prennent des décisions, et comment ils interagissent dans les marchés individuels.
    \pause
    \item \textbf{Macroéconomie}
    L'étude de l'économie dans son ensemble, incluant des mesures globales telles que
    le PIB, la consommation, l'investissement, l'inflation et le chômage.
    \begin{itemize}
        \item \textbf{Short-run(Court terme) :} Cycles économiques, récessions, et politiques monétaires et fiscales.\pause
        \item \textbf{Long-run(Long terme) :} Croissance économique, productivité et commerce international.
    \end{itemize}
    \item \textbf{Pourquoi étudier la macroéconomie ?}
\end{itemize}
\end{frame}


%------------------------------------------------
%  History of Macroeconomics
%------------------------------------------------
\begin{frame}
    \frametitle{Histoire de la Macroéconomie}
    \framesubtitle{Pré-critique de Lucas : 1936-1976}
        \begin{itemize}
            \item Le livre fondateur de John Maynard Keynes en 1936, \enquote{Théorie générale de l'emploi, de l'intérêt et de la monnaie}
            \item \textbf{Économie keynésienne :} Prône l'intervention gouvernementale pour stabiliser l'économie.
            \item \textbf{Limitations :} Basée sur des relations agrégées telles que la courbe de Phillips,
            une relation inverse entre l'inflation et le chômage \parencite{Phillips_1958}.
            \item Échec dans les années 1970 en raison de la stagflation, une combinaison d'inflation élevée et de chômage élevé,
            qui n'était pas expliquée par les modèles keynésiens.
        \end{itemize}
    \end{frame}
    \begin{frame}
        \frametitle{Histoire de la Macroéconomie}
        \framesubtitle{Post-critique de Lucas : 1976-Présent}
                \begin{itemize}
                    \item \textbf{La critique de Robert Lucas en 1976 :} Les micro-fondations sont essentielles pour les modèles macroéconomiques ! \parencite{Lucas_1976}.
                    \item A conduit au développement de la macroéconomie moderne, à commencer par la théorie du cycle économique réel \textcite{Kydland_Prescott_1982}.
                    \item \textbf{Principales réflexions :} Les attentes, la rationalité et les chocs.
                    \item \textbf{Économie néo-keynésienne :} Intègre les prix et les salaires rigides dans les modèles : modèles \textbf{DSGE}.
                \end{itemize}
    \end{frame}
                        
%------------------------------------------------
%  Economic Models
%------------------------------------------------
\begin{frame}
    \frametitle{Les Modèles Économiques}
    \framesubtitle{Qu'est-ce qu'un Modèle ?}
        \begin{itemize}
            \item Un modèle est une \textbf{représentation simplifiée} d'une réalité complexe.
            \item Les modèles nous aident à comprendre, expliquer et prédire les phénomènes économiques avec un cadre clair.
            \item \textbf{Objectif :} Abstraire le monde réel complexe en parties gérables.
        \end{itemize}
    \end{frame}
    \begin{frame}
        \frametitle{Les Modèles Économiques}
        \framesubtitle{Pourquoi Utiliser des Modèles ?}
        \begin{itemize}
            \item \textbf{Réalisation d'expériences:} Les modèles permettent aux économistes de conduire des expériences qui ne sont pas réalisables dans le monde réel.
            \item \textbf{Orientation des politiques :} Les résultats de ces expériences peuvent guider les décisions en matière de politique économique.
            \item \textbf{Outils exploratoires :} Ils aident à explorer les résultats de différents scénarios et politiques économiques.
            \item \textbf{Tous les modèles sont faux, mais certains sont utiles.}
            \item Les meilleurs modèles sont ceux qui offrent la plus grande clarté et puissance prédictive tout en reconnaissant leurs limites.
        \end{itemize}
    \end{frame}
    
%------------------------------------------------
%  GDP
%------------------------------------------------
\begin{frame}
    \frametitle{PIB}
    \framesubtitle{Définition et Composants}
        \begin{itemize}
            \item \textbf{PIB (Produit Intérieur Brut)} est la valeur marchande totale de tous les biens et services finaux produits à l'intérieur d'un pays pendant une période donnée.
            \item \textbf{Composants:}
                \begin{itemize}
                    \item \textbf{Consommation (C)} : Dépenses des ménages en biens et services.
                    \item \textbf{Investissement (I)} : Dépenses en biens de capital par les entreprises et les ménages.
                    \item \textbf{Dépenses Gouvernementales (G)} : Dépenses en biens et services par le gouvernement.
                    \item \textbf{Exportations Nettes (NX)} : Exportations moins importations.
                \end{itemize}
        \end{itemize}
\end{frame}
                
\begin{frame}
    \frametitle{PIB}
    \framesubtitle{Méthodes de Mesure}
        \begin{itemize}
            \item \textbf{Approche par la dépense :} Somme de toutes les dépenses effectuées dans l'économie :
                \begin{equation}
                PIB = C + I + G + (X - IM)
                \end{equation}
            \item \textbf{Approche du revenu :} Somme de tous les revenus perçus dans l'économie :
                \begin{equation}
                PIB = \text{Salaires} + \text{Loyers} + \text{Intérêts} + \text{Profits} + \text{Taxes} - \text{Subventions}
                \end{equation}
        \end{itemize}
\end{frame}

\begin{frame}
    \frametitle{Kaldor's Stylized Facts}
    \framesubtitle{Aperçus Clés sur la Croissance Économique}
        \begin{itemize}
            \item \textbf{Croissance de la Production :} La production par travailleur et la production totale ont augmenté de manière constante au fil du temps.
            \item \textbf{Accumulation de Capital :} Le stock de capital par travailleur augmente; cependant, le ratio capital-production reste relativement stable.
            \item \textbf{Taux de Rendement :} Le taux de rendement sur l'investissement reste stable malgré les augmentations significatives du stock de capital.
            \item \textbf{Répartition du Revenu :} Les parts du revenu national attribuées au travail et au capital restent relativement stables sur de longues périodes.
        \end{itemize}
\end{frame}
    
\end{document}
