\documentclass{beamer}
% Load csquotes for the \enquote command
\usepackage{csquotes}

% Load biblatex for bibliography management
\usepackage[backend=biber, style=authoryear]{biblatex}
\addbibresource{bibliography.bib} % Specify your BibTeX bibliography file

\mode<presentation> {
\usetheme{CambridgeUS}
\usecolortheme{seahorse}
}

\title{Introductory Macroeconomics for Engineers}
\author{Instructor Name}
\date{Semester and Year}

\begin{document}

\begin{frame}
\titlepage
\end{frame}

\begin{frame}
\frametitle{Overview}
\tableofcontents
\end{frame}

% Slide Template
\section{Introduction}
\begin{frame}
\frametitle{Introduction}
\begin{itemize}
    \item Course Overview
    \item Objectives
    \item Grading
\end{itemize}
\end{frame}

\section{Economic Concepts}
\begin{frame}
\frametitle{Micro vs. Macro}
\begin{itemize}
    \item \textbf{Microeconomics}
    The study of individual economic agents such as households and firms,
    how they make decisions, and how they interact in individual markets.
    \pause
    \item \textbf{Macroeconomics}
    The study of the economy as a whole, including aggregate measures such as
    GDP, consumption, investment, inflation, and unemployment.
    \begin{itemize}
        \item \textbf{Short-run:} Business cycles, recessions, and monetary and fiscal policy.
        \item \textbf{Long-run:} Economic growth, productivity, and international trade.
    \end{itemize}
    \item \textbf{Why study macroeconomics?}
\end{itemize}
\end{frame}


%------------------------------------------------
%  History of Macroeconomics
%------------------------------------------------
\begin{frame}
    \frametitle{History of Macroeconomics}
    \framesubtitle{Pre-Lucas Critique: 1936-1976}
        \begin{itemize}
            \item John Maynard Keynes's seminal book in 1936  \enquote{The General Theory of Employment, Interest, and Money}
            \item \textbf{Keynesian Economics:} Advocated for government intervention to stabilize the economy.
            \item \textbf{Limitations:} Based on aggregate relationships such as the Phillips Curve,
            an inverse relationship between inflation and unemployment \parencite{Phillips_1958}.
            \item Broke down in the 1970s due to stagflation, a combination of high inflation and high unemployment,
            which was unexplained by Keynesian models.
        \end{itemize}
    \end{frame}
    \begin{frame}
        \frametitle{History of Macroeconomics}
        \framesubtitle{Post-Lucas Critique: 1976-Present}
            \begin{itemize}
                \item \textbf{Robert Lucas's critique in 1976:} Micro-foundations are essential for macroeconomic models! \parencite{Lucas_1976}.
                \item Led to development of modern macro, starting with Real Business Cycle \textcite{Kydland_Prescott_1982}.
                \item \textbf{Key Insights:} Expectations, rationality and shocks.
                \item \textbf{New Keynesian Economics:} Incorporates sticky prices and wages into models: DSGE models.
            \end{itemize}
        \end{frame}
                    
%------------------------------------------------
%  Economic Models
%------------------------------------------------
\begin{frame}
    \frametitle{Understanding Economic Models}
    \framesubtitle{What is a Model?}
        \begin{itemize}
            \item A model is a \textbf{simplified representation} of a complex reality.
            \item Models help us understand, explain, and predict economic phenomena with a clear framework.
            \item \textbf{Purpose:} To abstract the complex real-world into manageable parts.
        \end{itemize}
    \end{frame}

    \begin{frame}
        \frametitle{Understanding Economic Models}
        \framesubtitle{Why Use Models?}
            \begin{itemize}
                \item \textbf{Conducting Experiments:} Models allow economists to conduct experiments that are not feasible in the real world.
                \item \textbf{Informing Policy:} Results from these experiments can guide policy-making decisions.
                \item \textbf{Exploratory Tools:} They help in exploring the outcomes of different economic scenarios and policies.
            \end{itemize}
        \end{frame}
        
        \begin{frame}
            \frametitle{Understanding Economic Models}
            \framesubtitle{Testing Model Usefulness}
                \begin{itemize}
                    \item A model designed to explain phenomenon \textit{x} can be tested by its ability to explain \textit{y}, a related but untargeted phenomenon.
                    \item \textbf{Test of Usefulness:} Whether it can illuminate aspects it was not specifically designed to explain.
                    \item A model's inability to explain every aspect of reality is not necessarily a drawback.
                    \item \textbf{ All models are wrong, but some are useful.}
                    \item The best models are those that offer the greatest clarity and predictive power while acknowledging their limitations.
                    \end{itemize}
                \end{frame}
%------------------------------------------------
%  GDP
%------------------------------------------------
\begin{frame}
    \frametitle{Basic Concepts}
    \framesubtitle{GDP - Overview}
        \begin{itemize}
            \item \textbf{GDP (Gross Domestic Product)} measures the total value of all final goods and services produced within a country during a specific period.
            \item \textbf{Goods and Services:} "Goods" are tangible like shirts; "Services" are intangible like education.
            \item \textbf{Final Goods:} Only considers goods and services sold to end-users, excludes intermediate goods to avoid double counting.
            \item \textbf{Current Prices:} Values are based on prices during the period being measured.
            \item \textbf{Exclusions:} Does not account for home labor,  or illegal activities.
        \end{itemize}
    \end{frame}
    
    \begin{frame}
        \frametitle{Basic Concepts}
        \framesubtitle{GDP - Expenditure vs Income Approach}
            \begin{itemize}
                \item GDP calculated as the sum of Consumption (C), Investment (I), Government Expenditures (G), and Net Exports (NX).
                \item \textbf{Net Exports (NX)} = Exports (X) - Imports (IM).
                \item \textbf{Why subtract imports?} We exclude imports because they are products produced outside the domestic economy.
                \item \textbf{Formula:} $GDP_t = C_t + I_t + G_t + (X_t - IM_t)$
            \end{itemize}
        \end{frame}

        \begin{frame}
            \frametitle{Basic Concepts}
            \framesubtitle{GDP - Income Approach}
                \begin{itemize}
                    \item The income approach calculates GDP by summing all incomes earned in the production of goods and services.
                    \item \textbf{Components:} Includes wages (labor income), rents (income from property), interest (income from capital), and profits (corporate earnings).
                    \item \textbf{Formula:} $GDP = \text{wages} + \text{rents} + \text{interest} + \text{profits} + \text{taxes on production and imports} - \text{subsidies}$
                    \item This method mirrors the expenditure approach as every dollar spent in an economy is a dollar income to someone else.
                    \item \textbf{Key Insight:} Helps in understanding the distribution of income in the economy, showing how much income is generated from various economic activities.
                \end{itemize}
            \end{frame}
            
        
        \begin{frame}
            \frametitle{Basic Concepts}
            \framesubtitle{Real vs. Nominal GDP}
                \begin{itemize}
                    \item \textbf{Nominal GDP:} Measured in current prices, reflects price and quantity changes.
                    \item \textbf{Real GDP:} Adjusted for price changes, provides a clearer measure of economic performance.
                    \item \textbf{Example:} If 10 units of a good are produced at a price of \$1.50 each, Nominal GDP is \$15.00.
                    \item \textbf{Calculating Real GDP:} Real GDP = $\frac{\text{Nominal GDP}}{\text{Price Level}}$ focuses on quantity only.
                \end{itemize}
            \end{frame}
            
\section{Economic Growth}
\begin{frame}
\frametitle{Kaldor's Stylized Facts}
\begin{itemize}
    \item Kaldor's stylized facts are a set of empirical regularities observed in economic growth.
    \item They were identified by economist Nicholas Kaldor in the 1950s and 1960s.
    \item These facts provide insights into the patterns and characteristics of economic growth.
    \item The stylized facts include:
    \begin{itemize}
        \item High and sustained rates of economic growth are possible.
        \item Economic growth is uneven across countries and regions.
        \item The distribution of income becomes more unequal during the early stages of economic growth.
        \item The share of labor income in national income tends to decline over time.
        \item The share of investment in national income tends to increase over time.
    \end{itemize}
\end{itemize}
\end{frame}
\end{document}
