\documentclass{beamer}

\mode<presentation> {
\usetheme{CambridgeUS}
\usecolortheme{seahorse}
}

\title{Introductory Macroeconomics for Engineers}
\author{Instructor Name}
\date{Semester and Year}

\begin{document}

\begin{frame}
\titlepage
\end{frame}

\begin{frame}
\frametitle{Overview}
\tableofcontents
\end{frame}

% Slide Template
\section{Introduction}
\begin{frame}
\frametitle{Introduction}
\begin{itemize}
    \item Course Overview
    \item Objectives
    \item Grading
\end{itemize}
\end{frame}

\section{Economic Concepts}
\begin{frame}
\frametitle{Micro vs. Macro}
\begin{itemize}
    \item Microeconomics
    The study of individual economic agents such as households and firms,
    how they make decisions, and how they interact in individual markets.
    \pause
    \item Macroeconomics
    The study of the economy as a whole, including aggregate measures such as
    GDP, consumption, investment, inflation, and unemployment.
    Short-run: Business cycles, recessions, and monetary and fiscal policy.
    Long-run: Economic growth, productivity, and international trade. \pause
    \item \textbf{Why study macroeconomics?}
\end{itemize}
\end{frame}

%------------------------------------------------
%  GDP
%------------------------------------------------
\begin{frame}
    \frametitle{Basic Concepts}
    \framesubtitle{GDP - Overview}
        \begin{itemize}
            \item \textbf{GDP (Gross Domestic Product)} measures the total value of all final goods and services produced within a country during a specific period.
            \item \textbf{Goods and Services:} "Goods" are tangible like shirts; "Services" are intangible like education.
            \item \textbf{Final Goods:} Only considers goods and services sold to end-users, excludes intermediate goods to avoid double counting.
            \item \textbf{Current Prices:} Values are based on prices during the period being measured.
            \item \textbf{Exclusions:} Does not account for home labor,  or illegal activities.
        \end{itemize}
    \end{frame}
    
    \begin{frame}
        \frametitle{Basic Concepts}
        \framesubtitle{GDP - Expenditure Approach}
            \begin{itemize}
                \item GDP calculated as the sum of Consumption (C), Investment (I), Government Expenditures (G), and Net Exports (NX).
                \item \textbf{Net Exports (NX)} = Exports (X) - Imports (IM).
                \item \textbf{Why subtract imports?} We exclude imports because they are products produced outside the domestic economy.
                \item \textbf{Formula:} $GDP_t = C_t + I_t + G_t + (X_t - IM_t)$
            \end{itemize}
        \end{frame}
        
        \begin{frame}
            \frametitle{Basic Concepts}
            \framesubtitle{Real vs. Nominal GDP}
                \begin{itemize}
                    \item \textbf{Nominal GDP:} Measured in current prices, reflects price and quantity changes.
                    \item \textbf{Real GDP:} Adjusted for price changes, provides a clearer measure of economic performance.
                    \item \textbf{Example:} If 10 units of a good are produced at a price of \$1.50 each, Nominal GDP is \$15.00.
                    \item \textbf{Calculating Real GDP:} Real GDP = $\frac{\text{Nominal GDP}}{\text{Price Level}}$ focuses on quantity only.
                \end{itemize}
            \end{frame}
            
%------------------------------------------------
%  Economic Models
%------------------------------------------------
\begin{frame}
    \frametitle{Understanding Economic Models}
    \framesubtitle{What is a Model?}
        \begin{itemize}
            \item A model is a \textbf{simplified representation} of a complex reality.
            \item Models help us understand, explain, and predict economic phenomena with a clear framework.
            \item \textbf{Purpose:} To abstract the complex real-world into manageable parts.
        \end{itemize}
    \end{frame}

    \begin{frame}
        \frametitle{Understanding Economic Models}
        \framesubtitle{Why Use Models?}
            \begin{itemize}
                \item \textbf{Conducting Experiments:} Models allow economists to conduct experiments that are not feasible in the real world.
                \item \textbf{Informing Policy:} Results from these experiments can guide policy-making decisions.
                \item \textbf{Exploratory Tools:} They help in exploring the outcomes of different economic scenarios and policies.
            \end{itemize}
        \end{frame}
        
        \begin{frame}
            \frametitle{Understanding Economic Models}
            \framesubtitle{Testing Model Usefulness}
                \begin{itemize}
                    \item A model designed to explain phenomenon \textit{x} can be tested by its ability to explain \textit{y}, a related but untargeted phenomenon.
                    \item \textbf{Test of Usefulness:} Whether it can illuminate aspects it was not specifically designed to explain.
                    \item A model's inability to explain every aspect of reality is not necessarily a drawback.
                \end{itemize}
            \end{frame}

            \begin{frame}
                \frametitle{Understanding Economic Models}
                \framesubtitle{All Models are Wrong, But Some are Useful}
                    \begin{itemize}
                        \item \textbf{Famous Adage:} All models are wrong, but some are useful.
                        \item This phrase emphasizes that while no model can capture all aspects of reality, many provide significant insights and practical value.
                        \item \textbf{Utility:} The best models are those that offer the greatest clarity and predictive power while acknowledging their limitations.
                    \end{itemize}
                \end{frame}

%------------------------------------------------
%  History of Macroeconomics
%------------------------------------------------
\begin{frame}
    \frametitle{History of Macroeconomics}
    \framesubtitle{The Early Period: 1936-1968}
        \begin{itemize}
            \item John Maynard Keynes published his seminal book during the Great Depression in 1936, sparking vast debates on his theories.
            \item John Hicks offered a graphical interpretation in 1937, popularizing Keynesian economic models for policy.
            \item Advancements in computational power in the 1950s allowed for the creation of complex statistical models by economists like Lawrence Klein, focusing on forecasting economic trends.
            \item The Klein and Goldberger model (1955) integrated dozens of equations, enabling dynamic predictions of economic responses to shocks.
        \end{itemize}
    \end{frame}
    
    \begin{frame}
        \frametitle{History of Macroeconomics}
        \framesubtitle{The Phillips Curve and its Implications}
            \begin{itemize}
                \item The Phillips Curve, introduced in 1958, illustrated a robust inverse relationship between inflation and unemployment rates.
                \item This relationship suggested that policymakers could manipulate monetary and fiscal policies to target specific inflation and unemployment rates.
                \item Milton Friedman critiqued this in 1968, arguing that such manipulation could lead to an inflationary spiral, challenging the sustainability of the Phillips Curve.
            \end{itemize}
        \end{frame}
                        
        \begin{frame}
            \frametitle{History of Macroeconomics}
            \framesubtitle{Breakdown Era: 1968-1981}
                \begin{itemize}
                    \item The era highlighted the lack of microeconomic foundations in early Keynesian models.
                    \item Robert Lucas's critique (1976) emphasized that models based on ad hoc macroeconomic relationships fail under policy changes due to the rational expectations of agents.
                    \item This period saw the dismissal of simplistic Keynesian models, influenced further by economic stagnation during the 1970s.
                \end{itemize}
            \end{frame}
            
            \begin{frame}
                \frametitle{History of Macroeconomics}
                \framesubtitle{Modern Macroeconomics: 1982-Present}
                    \begin{itemize}
                        \item Introduction of Real Business Cycle theory by Kydland and Prescott in 1982, integrating microeconomic behaviors into macro models.
                        \item Their model demonstrated how random technology shocks could drive business cycles, challenging earlier Keynesian views.
                        \item Development of DSGE models that integrate rational expectations, optimizing behaviors, and acknowledge various market frictions.
                        \item Current models blend insights from Keynesian approaches with new microeconomic foundations, shaping modern macroeconomic policy tools.
                    \end{itemize}
            \end{frame}
                

\end{document}
