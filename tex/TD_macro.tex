\documentclass[11pt]{article}
\usepackage{geometry}
\geometry{letterpaper}
\usepackage{amsmath} % for mathematical features
\usepackage{enumitem} % for customizing lists
\title{Introductory Macroeconomics for Engineers}
\author{Martin A. Valdez}
\date{IE 1}

\begin{document}
\maketitle

\section*{TD Macro 1}

\subsection*{Exercise 1}
Express the following equations as log-linear functions, i.e., take logs and simplify:
\begin{enumerate}
    \item[(a)] $Y = zK^\alpha N^{1-\alpha}$.
    \item[(b)] $Z = ce^{rt^\beta K}$.
\end{enumerate}

\subsection*{Exercise 2}
Calculate the first and second derivatives of the following functions:
\begin{enumerate}
    \item[(a)] $f(c) = \ln(c)$.
    \item[(b)] $u(c) = \frac{c^{1-\sigma}}{1-\sigma}$.
    \item[(c)] $h(w) = (-6w^3 + 17w - 4)^\beta - \ln(\theta w^\beta)$.
\end{enumerate}

\subsection*{Exercise 3}
Calculate all the first, second, and cross derivatives of the following functions:
\begin{enumerate}
    \item[(a)] $F(K, N) = \theta K^\alpha N^{1-\alpha}$.
    \item[(b)] $F(K, N) = \ln \theta + \alpha \ln K + (1 - \alpha) \ln N$.
\end{enumerate}

\subsection*{Exercise 4}
Solve the following constrained maximization problem. 
\[
\max_{x,w} U = \alpha \ln(x) + \beta \ln(w) 
\]
subject to
\begin{align*}
    &p_x x + p_w w \leq y 
\\
&\alpha + \beta = 1.
\end{align*}

\subsection*{Exercise 5}
Consider the function \( f(x) = \ln(1+x) \). 
Calculate the first-order Taylor expansion of \( f(x) \) around the point \( x = 0 \).
Show that a growth rate can be approximated by the first-order Taylor expansion 
of the logarithm function around the point 1.

\subsubsection*{Note on First Order Taylor Expansion}
The first-order Taylor expansion of a function \( f(x) \) around a point \( a \) provides a linear approximation of \( f(x) \) near \( a \). It is given by:
\[
f(x) \big|_{x=a} \approx f(a) + f'(a)(x - a),
\]
where \( f'(a) \) is the derivative of \( f \) at \( a \).

% \textbf{Solution:}
% \begin{align*}
%     f(x) &= \ln(1+x)
% \\
%     f'(x) &= \frac{1}{1+x}
% \\
%     f(x) \big|_{x=0} &\approx f(0) + f'(0)(x - 0)
% \\
%     \ln(1+x) \big|_{x=0} &\approx \ln(1) + \frac{1}{1}(x) = 0 + x = x
%    \\ \text{Decompose the growth rate:}&
% \\
%     g_x(t) = \frac{x_t - x_{t-1}}{x_{t-1}} &= \frac{x_t}{x_{t-1}} - 1 
%     \implies
%     1 + g_x(t) = \frac{x_t}{x_{t-1}}
% \\
%     \ln(1 + g_x(t)) &\approx  g_x(t) = \ln\left(\frac{x_t}{x_{t-1}}\right) = \ln(x_t) - \ln(x_{t-1})
%     \\ 
%     g_x(t) &\approx \Delta \ln(x_t)
% \end{align*}




\subsection*{Exercise 6}
Suppose an economy produces steel, wheat, and oil. Here are the economic activities of each industry:
\begin{itemize}
    \item The steel industry produces \$100,000 in revenue, spends \$4,000 on oil, \$10,000 on wheat, and pays workers \$80,000.
    \item The wheat industry produces \$150,000 in revenue, spends \$20,000 on oil, \$10,000 on steel, and pays workers \$90,000.
    \item The oil industry produces \$200,000 in revenue, spends \$40,000 on wheat, \$30,000 on steel, and pays workers \$100,000.
\end{itemize}
There is no government, and there are neither exports nor imports. None of the industries accumulate or deaccumulate inventories.
\begin{enumerate}
    \item Calculate the GDP of this economy using the production method.
    \item Calculate the GDP using the income method.
\end{enumerate}

\subsection*{Exercise 7}
Excel exercise: Download file data\_td\_1.xlsx from the moodle.
\begin{enumerate}
    \item Generate a series for the natural logarithm of realGDP and realGDP per capita for both countries.
    \item Use the series for the natural logarithm of gdp to calculate the growth rate of real gdp and
            real gdp per capita for both countries.
    \item Compute the average growth rate of real gdp and real gdp per capita for both countries.
    \item Plot the evolution of realGDP, realGDP per capita, and the growth rate of realGDP for both countries.
    How does the French economy compare to the economy of the United States?
\end{enumerate}
\subsection*{Exercise 8}
Consider the following Cobb-Douglas production function:
\begin{align*}
    Y_t &= A_t K_t^\alpha L_t^{1-\alpha},
\end{align*}
with total factor productivity represented by \( A_t \). 
Derive the expression for growth rates using the logarithmic form of the production function. Show that the growth rate of output (\( g_{Y_t} \)) can be decomposed into the weighted sum of the growth rates of capital (\( g_{K_t} \)), labor (\( g_{L_t} \)), and total factor productivity (\( g_{A_t} \)):
\begin{align*}
    g_{Y_t} = \alpha g_{K_t} + (1-\alpha) g_{L_t} + g_{A_t}.    
\end{align*}

\subsection*{Exercise 9}
Maximize the profits of a representative competitive firm that 
produces the economy's total output using constant returns to scale 
Cobb-Douglas technology, paying wages \( w_t \) and renting capital at rate \( r_t \). 
Demonstrate that the weights for capital and labor growth rates, 
from the previous exercise, 
are the capital's and labor's shares of income.
Hint: The firm's profit maximization problem is:
\begin{align*}
    \max_{K_t, L_t} \Pi_t = A_t K_t^\alpha L_t^{1-\alpha} - w_t L_t - r_t K_t.
\end{align*}


\subsection*{Exercise 10}
Download the data\_td\_2.xlsx file from Moodle. 
Follow these steps to compute the labor income share, 
capital income share, and the Solow residual, which is 
defined as the growth rate of productivity.

\begin{enumerate}
    \item Calculate labor and capital income shares as the ratio of total labor 
    income and total capital income to total output, respectively.

    \item Calculate the growth rates of output, labor, and capital
    using the definition of the growth rate.
    \item Compute the Solow residual as the difference between the growth rate of
    output and the weighted sum of the growth rates of labor and capital.
    \item Now do the same using logarithms to compute the growth rates.
    \item Plot both solow residuals.
\end{enumerate}





\end{document}
