
\documentclass[11pt]{article}
\usepackage{geometry}
\geometry{letterpaper}
\usepackage{amsmath} % for mathematical features
\usepackage{enumitem} % for customizing lists
\title{Introductory Macroeconomics for Engineers}
\author{Martin A. Valdez}
\date{IE 1}

\begin{document}
\maketitle
\subsection*{Exercice 7}
Exercice Excel : Téléchargez le fichier data\_td\_1.xlsx depuis Moodle.
\begin{enumerate}
    \item Générez une série pour le logarithme naturel du PIB réel et du PIB réel par habitant pour les deux pays.
    \item Utilisez la série pour le logarithme naturel du PIB pour calculer le taux de croissance du PIB réel et
            du PIB réel par habitant pour les deux pays.
    \item Calculez le taux de croissance moyen du PIB réel et du PIB réel par habitant pour les deux pays.
    \item Tracez l'évolution du PIB réel, du PIB réel par habitant, et du taux de croissance du PIB réel pour les deux pays.
    Comment l'économie française se compare-t-elle à l'économie des États-Unis ?
\end{enumerate}
\subsection*{Exercice 8}
Considérez la fonction de production de Cobb-Douglas suivante :
\begin{align*}
    Y_t &= A_t K_t^\alpha L_t^{1-\alpha},
\end{align*}
où la productivité totale des facteurs est représentée par \( A_t \). 
Dérivez l'expression des taux de croissance en utilisant la forme logarithmique de la fonction de production. Montrez que le taux de croissance de la production (\( g_{Y_t} \)) peut être décomposé en une somme pondérée des taux de croissance du capital (\( g_{K_t} \)), du travail (\( g_{L_t} \)), et de la productivité totale des facteurs (\( g_{A_t} \)):
\begin{align*}
    g_{Y_t} = \alpha g_{K_t} + (1-\alpha) g_{L_t} + g_{A_t}.    
\end{align*}

\subsection*{Exercice 9}
Maximisez les profits d'une entreprise représentative compétitive qui
produit la production totale de l'économie en utilisant une technologie Cobb-Douglas à rendements constants d'échelle, en payant des salaires \( w_t \) et en louant du capital au taux \( r_t \).
Démontrez que les poids des taux de croissance du capital et du travail,
de l'exercice précédent,
sont les parts de revenu du capital et du travail.
Indice : Le problème de maximisation du profit de l'entreprise est :
\begin{align*}
    \max_{K_t, L_t} \Pi_t = A_t K_t^\alpha L_t^{1-\alpha} - w_t L_t - r_t K_t.
\end{align*}

\subsection*{Exercice 10}
Téléchargez le fichier data\_td\_2.xlsx depuis Moodle.
Suivez ces étapes pour calculer la part du revenu du travail,
la part du revenu du capital, et le résidu de Solow, qui est
défini comme le taux de croissance de la productivité.

\begin{enumerate}
    \item Calculez les parts du revenu du travail et du capital comme le rapport du revenu total du travail et du revenu total du capital au produit total, respectivement.
    \item Calculez les taux de croissance de la production, du travail et du capital
    en utilisant la définition du taux de croissance.
    \item Calculez le résidu de Solow comme la différence entre le taux de croissance de la production et la somme pondérée des taux de croissance du travail et du capital.
    \item Faites de même en utilisant des logarithmes pour calculer les taux de croissance.
    \item Tracez les deux résidus de Solow.
\end{enumerate}


\end{document}