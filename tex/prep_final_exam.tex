
\documentclass[11pt]{article}
\usepackage{geometry}
\geometry{letterpaper}
\usepackage{amsmath} % for mathematical features
\usepackage{enumitem} % for customizing lists
\title{Introductory Macroeconomics for Engineers}
\author{Martin A. Valdez}
\date{IE 1\\
Final Exam Preparation}

\begin{document}
\maketitle

\subsection*{Question 1}

\begin{enumerate}
    \item Why do we study macroeconomics? (0.5 pts)
    \item Why do we use models in macroeconomics?(0.5 pts)
\end{enumerate}

\subsection*{Question 2}
\begin{enumerate}
    \item What is the definition of GDP? (1 pt)
    \item What are the three approaches to measuring GDP? Provide the formula for each approach. (1.5 pt - 0.5 per approach)
    \item What is the difference between nominal and real GDP? (0.5 pt)
\end{enumerate}

\subsection*{Question 3}
Mention at least four of Kaldor's stylized facts. (0.5 pt each)

\subsection*{Question 4}
Suppose that total output in an economy is given by the following production function:
\begin{align*}
    Y = A F(K,N) = A K^{\alpha} L^{1-\alpha},
\end{align*}
Where \(Y\) is total output, \(K\) is capital, \(L\) is labor, \(A\) is total factor productivity, and \(\alpha \in (0,1)\) 
is the capital share of output.

\begin{enumerate}
    \item Show that F has constant returns to scale, i.e., that \(F(\lambda K, \lambda L) = \lambda F(K,L)\) for all \(\lambda > 0\). (1 pt)
    \item Compute \(\frac{\partial Y }{\partial K} \) and \(\frac{\partial Y }{\partial L} \). (1 pt each)
    \item Use the log transformation to compute the growth rate of output, \(g_Y\), in terms of the growth rates of capital, \(g_K\), labor, 
    \(g_L\), and total factor productivity, \(g_A\). (2 pt)
    \item Suppose that output from \(t\) to \(t+1\) grows 10\%. The capital stock grows from 50 to 60 euros, and the labor
    force grows from 10 to 12 workers. Suppose that \(\alpha = 0.3\).
    \begin{enumerate}
        \item Calculate the growth rate of capital and labor. (1 pt each)
        \item Calculate the growth rate of total factor productivity. (1 pt)
    \end{enumerate}
\end{enumerate}

\subsection*{Question 5}
Recall the Solow model, whose dynamics are given by the following equation:
\begin{align*}
    K_{t+1} = sA K_t^{\alpha}L_t^{1-\alpha} + (1-\delta)K_t
\end{align*}
Where \(K_t\) is the capital stock, \(L_t\) is the labor force, \(s\) is the savings rate, \(\delta\) is the depreciation rate, and \(A\) 
is total factor productivity.

\begin{enumerate}
    \item Transform the equation above into a per-worker basis. (2 pts)
    \item Graph the per-worker capital stock, \(k_{t+1}\), as a function of the per-worker capital stock, \(k_t\). (2 pt)
    \item Show the dynamics of the per-worker capital stock, \(k_{t+1}\) in the graph using the curve \(k_{t+1} = k_t\). (1 pt)
    \item Solve for the equilibrium level of capital per worker, \(k^*\). (2 pt)
    \item What is the growth rate of the per-worker capital stock in the steady state? (1 pt)
\end{enumerate}

\subsection{ Bonus Question}
Where does the savings rate parameter, \(s\), come from in the Solow model? (1 pt)

\end{document}