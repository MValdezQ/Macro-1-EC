\documentclass{beamer}
% Load csquotes for the \enquote command
\usepackage{csquotes}
\usepackage{hyperref}
% Load biblatex for bibliography management
\usepackage[backend=biber, style=authoryear]{biblatex}
\addbibresource{bibliography.bib} % Specify your BibTeX bibliography file

\mode<presentation> {
\usetheme{CambridgeUS}
\usecolortheme{seahorse}
}

\title{Macroéconomie 1}
\author{Mart\'in Valdez}
\date{IE1}

\begin{document}

\begin{frame}
\titlepage
\end{frame}

% \begin{frame}
% \frametitle{Overview}
% \tableofcontents
% \end{frame}

% ------------------------------------------------
%  Introduction
% ------------------------------------------------
\section{Modèle de Solow}
%-------------------------------------------------
%  First Macro Model: Solow Growth Model
%-------------------------------------------------

\begin{frame}
    \frametitle{Modèle de Solow}
    \framesubtitle{Introduction}
    \begin{itemize}
        \item Le modèle de Solow, développé par Solow en 1956, est utilisé pour étudier la croissance économique à long terme et les variations de revenu entre les pays.\pause
        \item \textbf{Implication principale :} 
        La productivité est \textbf{cruciale} pour la croissance économique soutenue et est 
        \textbf{plus} significative que l'accumulation de facteurs.\pause
        \item \textbf{Principaux inconvénients :}
        \begin{itemize}
            \item La productivité est considérée comme exogène.
            \item La consommation est supposée constante.
            \item Le modèle simplifie excessivement en ignorant des facteurs tels que 
            le capital humain, le progrès technologique, les imperfections du marché, 
            la diversité des agents, les rôles gouvernementaux, etc.
        \end{itemize}
    \end{itemize}
\end{frame}

\begin{frame}
    \frametitle{Modèle de Solow}
    \framesubtitle{Introduction}
    \begin{itemize}
        \item Le temps s'écoule de \( t \) (le présent) vers un futur infini.
        \item Modélise un ménage représentatif et une entreprise représentative.
        \item Considère un seul bien qui représente tout ce qui est réel dans l'économie.\pause
        \item \textbf{Fonction de production :} \( Y_t = A_t F(K_t, N_t) \)
        \begin{itemize}
            \item \( K_t \) : capital, qui est produit, utilisé pour fabriquer d'autres biens, et ne se déprécie pas complètement.
            \item \( N_t \) : travail, représentant le temps passé à utiliser les machines pour produire des biens.
            \item \( Y_t \) : production, que l'on peut considérer comme des unités de nourriture.
            \item \( A_t \) : productivité (exogène), affecte l'efficacité du capital et du travail.
        \end{itemize}
        \pause
        \item 
        Conceptualisez la production comme des \enquote{fruits}, le stock de capital comme des \enquote*{arbres fruitiers} 
        et le travail comme le temps passé à cultiver les arbres.
    \end{itemize}
\end{frame}

\begin{frame}
    \frametitle{Modèle de Solow}
    \framesubtitle{Fonction de production}
    \begin{itemize}
        \item Les deux entrées sont nécessaires : \( F(0, N_t) = F(K_t, 0) = 0 \).
        \item Augmentation avec les deux entrées : \( F_K(K_t, N_t) > 0 \) et \( F_N(K_t, N_t) > 0 \).
        \item Concavité dans les deux entrées : \( F_{KK}(K_t, N_t) < 0 \) et \( F_{NN}(K_t, N_t) < 0 \).
        \item Rendements constants à l'échelle : \( F(qK_t, qN_t) = qF(K_t, N_t) \).
        \item Le capital et le travail sont payés à leurs produits marginaux :
        \begin{itemize}
            \item \( w_t = A_t F_N(K_t, N_t) \) (taux salarial)
            \item \( R_t = A_t F_K(K_t, N_t) \) (rendement du capital)
        \end{itemize}
        (pourquoi ?)\pause
        \item Fonction de production exemple : Cobb-Douglas :
        \[ F(K_t, N_t) = K_t^\alpha N_t^{1-\alpha}, \quad 0 < \alpha < 1 \]
        \item La fonction de production est-elle réaliste ? Non ! \parencite{Banerjee_2005}.
        Alors pourquoi l'utilisons-nous ?
        
    \end{itemize}
\end{frame}

\begin{frame}
    \frametitle{Modèle de Solow}
    \framesubtitle{Consommation et Investissement}
    \begin{itemize}
        \item Les fruits peuvent être consommés (consommation) ou replantés dans le sol (investissement), 
        ce qui produit ensuite un autre arbre (capital) avec un délai d'un période.
        \item On suppose qu'une fraction constante de la production, 
        \( 0 \leq s \leq 1 \), est investie. 
        Ceci est le "taux d'épargne" ou "taux d'investissement." 
        (Plus de détails plus tard !)
        \pause
        \item \textbf{Contrainte de ressources :} \( Y_t = C_t + I_t \)
        (\enquote{Fermeture du modèle})\pause
        \item Équation d'accumulation du capital avec un taux de dépréciation \( 0 < \delta < 1 \) :
        \[ K_{t+1} = I_t + (1 - \delta)K_t \]
    \end{itemize}
\end{frame}


\begin{frame}
    \frametitle{Modèle de Solow}
    \framesubtitle{Équation Centrale et Dynamique}
    \begin{itemize}
        \item Équations simplifiées :
        \begin{align*}
            Y_t &= A_t F(K_t, N_t) \\
            C_t &= (1 - s)Y_t \\
            I_t &= sY_t \\
            w_t &= A_t F_N(K_t, N_t) \\
            R_t &= A_t F_K(K_t, N_t)
        \end{align*}
        \item Combinez les quatre premières équations en une seule équation dynamique centrale
        \pause
        \[ K_{t+1} = sA_t F(K_t, N_t) + (1 - \delta)K_t \]
        \item Définissez les variables par travailleur : \( k_t = \frac{K_t}{N_t} \)
        \item Dynamique par travailleur : \( k_{t+1} = sA_t f(k_t) + (1 - \delta)k_t \)
    \end{itemize}
\end{frame}

\begin{frame}
    \frametitle{Modèle de Solow}
    \framesubtitle{L'état Stationnaire}
    \begin{itemize}
        \item Le stock de capital à l'état stationnaire, \( k^* \), est là où \( k_{t+1} = k_t \).
        \item Graphiquement, c'est là où la courbe de \( k_{t+1} \) croise la ligne à 45 degrés.
        \item Sous les hypothèses de la fonction de production et des conditions d'Inada, il existe un stock de capital à l'état stationnaire non nul.
        \item Stabilité : Pour toute valeur initiale \( k_t \neq 0 \), le stock de capital converge vers ce point.
        \item Implications : Une fois le capital atteint \( k^* \), toutes les autres variables se stabilisent également à leurs valeurs à l'état stationnaire, régies par \( k^* \).
        \item Exemple avec Cobb-Douglas : \( f(k_t) = k_t^\alpha \)
    \end{itemize}
\end{frame}


\end{document}
