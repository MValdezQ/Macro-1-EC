
\documentclass[11pt]{article}
\usepackage{geometry}
\geometry{letterpaper}
\usepackage{amsmath} % for mathematical features
\usepackage{enumitem} % for customizing lists
\usepackage{csquotes}
\title{Introductory Macroeconomics}
\author{Martin A. Valdez}

\date{IE 1\\}

\begin{document}

% \maketitle

\section*{Examen Final d'Introduction à la Macroéconomie}
\textbf{Lisez attentivement l'examen et essayez de répondre à toutes les questions du mieux que vous 
pouvez. 
Vous êtes autorisé à utiliser une calculatrice. Bon courage!}


\begin{flushleft}

Nom, Prenom:\underline{\hspace{5cm}}\hfil Nr. d'étudiant:\underline{\hspace{4cm}}\\
Groupe TD:\underline{\hspace{5cm}} \hfill Date:\underline{\hspace{5cm}}


\end{flushleft}
\subsection*{Question 1}

\begin{enumerate}
    \item Qu'est-ce que la macroéconomie ? (0.5 pt)
    \item Pourquoi étudions-nous la macroéconomie ? ( 0.5 pt)
    \item Quel est l'intérêt d'utiliser des modèles en macroéconomie ? (0.5 pt)
    \item Quelle est la différence entre la macroéconomie et la microéconomie ? (0.5 pt)
\end{enumerate}

\subsection*{Question 2}
\begin{enumerate}
    \item Quelle est la définition du PIB ? (1 pt)
    \item Quelles sont les trois approches pour mesurer le PIB ? Fournissez la formule pour chaque approche. (1.5 pt - 0.5 par approche)
    \item Quelle est la différence entre le PIB nominal et le PIB réel ? (1 pt)
\end{enumerate}

\subsection*{Question 3}
Mentionnez au moins 5 des faits stylisés de Kaldor. (0,5 pt chacun)

\subsection*{Question 4}
Supposons que nous sommes dans une économie où le PIB total est produit en utilisant la fonction de production suivante:
\begin{align*}
    Y = A F(K,N) = A K^{\alpha} L^{1-\alpha},
\end{align*}
Où \(Y\) est le PIB total, \(K\) est le stock total de capital dans l'économie, \(L\) est la force de travail de l'économie, qui est égale à la population (il n'y a pas de chômage dans cette économie), \(A\) est un paramètre de productivité appelé "Productivité Totale des Facteurs", et \(\alpha \in (0,1)\) est un paramètre qui contrôle la part de la production qui va au capital.

\begin{enumerate}
    \item Montrez que F est une fonction homogène de degré 1, également appelée rendements d'échelle constants, c'est-à-dire que \(F(\lambda K, \lambda L) = \lambda F(K,L)\) pour tout \(\lambda > 0\). (1 pt)
    \item Calculez les deux dérivées partielles de F, \(F_K = \frac{\partial F }{\partial K} \) et \(F_{L} = \frac{\partial F }{\partial L} \). (1 pt chacune)
    \item Utilisez une transformation logarithmique pour obtenir une expression du taux de croissance du PIB, \(g_Y\), en termes des taux de croissance du capital, \(g_K\), du travail, \(g_L\), et de la productivité totale des facteurs, \(g_A\). (2 pt)
    \item Supposons que le PIB (Y) augmente de 10\% de \(t\) à \(t+1\). Le stock de capital passe de 50 à 60 (en termes de sa valeur en euros), et la force de travail passe de 10 à 12 travailleurs. Supposons que \(\alpha = 0.3\) et qu'il n'y a ni inflation ni chômage dans cette économie.
    \begin{enumerate}
        \item Calculez le taux de croissance du capital \(g_K\) et du travail \(g_L\) en utilisant la définition du taux de croissance. (1 pt chacun)
        \item Calculez le taux de croissance de la productivité totale des facteurs \(g_A\) en utilisant l'expression du taux de croissance du PIB \(g_Y\) de la question 3, ainsi que les taux de croissance du capital et du travail que vous venez de calculer. Montrez votre raisonnement (1 pt)
    \end{enumerate}
\end{enumerate}

\subsection*{Question 5}
Rappelez-vous le modèle de Solow, dont la dynamique est donnée par les équations suivantes:
\begin{align}
    &K_{t+1} =I_t + (1-\delta)K_t,
    \ 
    Y_t = A K_t^{\alpha} L_t^{1-\alpha},
    \ Y_t = C_t + I_t,
    \ I_t = sY_t
\end{align}
Où \(K_t\) est le stock de capital, \(L_t\) est la force de travail, \(s\) est le taux d'épargne, \(\delta\) est le taux de dépréciation, et \(A\) 
est la productivité totale des facteurs. Rappelons qu'une variable par travailleur \(x_t\) est définie comme \(x_t = \frac{X_t}{L_t}\), 
où \(X_t\) est la variable agrégée.

\begin{enumerate}
    \item Tracez le graphique de l'équation précédente dans un graphique où l'axe des x est \(k_t\),
    le capital par travailleur au temps \(t\), et l'axe des y est \(k_{t+1}\),
    le capital par travailleur au temps \(t+1\), et montrez graphiquement le point d'équilibre du système.
    Montrez les dynamiques du capital par travailleur en utilisant la courbe \(k_{t+1} = k_t\).
    (1 pt)
    \item Résoudre pour le niveau d'état stable du capital par travailleur, \(k^*\), en termes des paramètres du modèle
    (\(s\), \(\delta\), \(A\) et \(\alpha\)). (1 pt)
    \item Quelle est le taux de croissance du capital par travailleur dans l'état stable, \(k^*\)? (1 pt)
    \item Suppose we are in the steady state level of capital per worker, \(k^*\), at
    time \(t\), and then the total factor productivity level, \(A\), increases from
    \(A \to A'>A\) that same period.  
    le niveau de productivité totale des facteurs, \(A\), augmente de 
    \(A \to A'>A\) au temps \(t\). 
    \begin{enumerate}
        \item Comment cela affectera-t-il le niveau d'état stable du capital par travailleur, \(k^*\)? (1 pt)
        \item Comment cela affectera-t-il le taux de croissance du capital par travailleur dans l'état stable, \(k^*\)? (1 pt)
    \end{enumerate}
    \item Comment on pourrait augmenter le modèle pour avoir une croissance économique soutenue? (1 pt)
\end{enumerate}

\end{document}