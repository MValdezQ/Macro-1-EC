\documentclass[11pt]{article}
\usepackage{geometry}
\geometry{letterpaper}
\usepackage{amsmath} % for mathematical features
\usepackage{enumitem} % for customizing lists
\title{Introductory Macroeconomics for Engineers}
\author{Martin A. Valdez}
\date{IE 1}

\begin{document}
\maketitle

\section*{TD Macro 1}

\subsection*{Exercice 1}
Exprimez les équations suivantes sous forme log-linéaire, c'est-à-dire prenez les logarithmes et simplifiez :
\begin{enumerate}
    \item[(a)] \(Y = zK^\alpha N^{1-\alpha}\).
    \item[(b)] \(Z = ce^{rt^\beta K}\).
\end{enumerate}

\subsection*{Exercice 2}
Calculez les premières et secondes dérivées des fonctions suivantes :
\begin{enumerate}
    \item[(a)] \(f(c) = \ln(c)\).
    \item[(b)] \(u(c) = \frac{c^{1-\sigma}}{1-\sigma}\).
    \item[(c)] \(h(w) = (-6w^3 + 17w - 4)^\beta - \ln(\theta w^\beta)\).
\end{enumerate}

\subsection*{Exercice 3}
Calculez toutes les premières, secondes et dérivées croisées des fonctions suivantes :
\begin{enumerate}
    \item[(a)] \(F(K, N) = \theta K^\alpha N^{1-\alpha}\).
    \item[(b)] \(F(K, N) = \ln \theta + \alpha \ln K + (1 - \alpha) \ln N\).
\end{enumerate}

\subsection*{Exercice 4}
Résolvez le problème de maximisation suivant sous contrainte.
\[
\max_{x,w} U = \alpha \ln(x) + \beta \ln(w) 
\]
sous la contrainte
\begin{align*}
    &p_x x + p_w w \leq y \\
    &\alpha + \beta = 1.
\end{align*}

\subsection*{Exercice 5}
Considérez la fonction \( f(x) = \ln(1+x) \). 
Calculez le Taylor expansion du premier ordre de \( f(x) \) autour du point \( x = 0 \).
Montrez que le taux de croissance peut être approché par le  Taylor expansion du premier 
ordre de la fonction logarithme autour du point 1, quand \( x \) est petit.

\subsubsection*{Note sur Taylor expansions du premier ordre}
Le Taylor expansion du premier ordre d'une fonction \( f(x) \) autour d'un point \( a \) fournit une approximation linéaire de \( f(x) \) près de \( a \). 
Il est donné par :
\[
f(x) \big|_{x=a} \approx f(a) + f'(a)(x - a),
\]
où \( f'(a) \) est la dérivée de \( f \) en \( a \).

\subsection*{Exercice 6}
Supposez qu'une économie produit de l'acier, du blé et du pétrole. Voici les activités économiques de chaque industrie :
\begin{itemize}
    \item L'industrie de l'acier produit 100 000 \$ de revenus, dépense 4 000 \$ en pétrole, 10 000 \$ en blé et paie 80 000 \$ aux travailleurs.
    \item L'industrie du blé produit 150 000 \$ de revenus, dépense 20 000 \$ en pétrole, 10 000 \$ en acier et paie 90 000 \$ aux travailleurs.
    \item L'industrie du pétrole produit 200 000 \$ de revenus, dépense 40 000 \$ en blé, 30 000 \$ en acier et paie 100 000 \$ aux travailleurs.
\end{itemize}
Il n'y a ni gouvernement, ni exportations, ni importations. Aucune des industries n'accumule ni ne déstocke des inventaires.
\begin{enumerate}
    \item Calculez le PIB de cette économie en utilisant la méthode de production.
    \item Calculez le PIB en utilisant la méthode du revenu.
\end{enumerate}


\end{document}
